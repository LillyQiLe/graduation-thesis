%% google scholar GB/T 7714

\begin{thebibliography}{2}
\setlength{\baselineskip}{25pt}

\bibitem{TTN}谭铁牛. 人工智能的历史、现状和未来[J]. 智慧中国, 2019: 87-91.

\bibitem{LXLL}
李仁发,谢勇,李蕊等. 信息$\!$-物理融合系统若干关键问题综述[J]. 计算机研究与发展, 2012, 49(6): 1149-1161.

\bibitem{Hjf}何积丰. Cyber-Physical Systems[J]. 中国计算机学会通讯, 2010, 6(1): 25–29.

\bibitem{LZZ}黎作鹏,张天驰,张菁. 信息物理融合系统(CPS)研究综述[J]. 计算机科学, 2011, 38(9): 25-31.

\bibitem{LFJWQ}李馥娟,王群,钱焕延. 信息物理融合系统研究[J]. 电子技术应用, 2018, 44(09): 30–34.

\bibitem{ZYW}朱艺伟. 信息物理融合系统的时空大数据的建模与分析[D]. 广东工业大学, 2019. 

\bibitem{CEM}CLARKE E M, WING J M. Formal Methods: State of the Art and Future Directions[M]. ACM, 1996: 626–643.

\bibitem{WSZ}王士忠. 参数化时空规范语言的语义模型及其应用[D]. 华东师范大学, 2018.

\bibitem{Chen}CHEN Y. STeC: A Location-Triggered Specification Language for Real-Time Systems[C] //IEEE International Symposium on Object/component/service-Oriented Real-Time Distributed Computing Workshops. IEEE, 2012: 1-6.

\bibitem{ChenY}CHEN Y, ZHANG Y. A Hybrid Clock System Related to STeC Language[C] //IEEE Eighth International Conference on Software Security and Reliability-Companion. IEEE, 2014: 199-203.

\bibitem{AE}ALLEN E E, LEI C L. Modalities for Model Checking: Branching Time Logic Strikes Back[J]. Science of Computer Programming, 1987, 8(3): 275-306. 

\bibitem{JZ}纪政. 基于STeC的时空一致性智能体建模与调控仿真验证[D]. 华东师范大学, 2014.

\bibitem{LXu}陆旭. 时空逻辑$\rm PPTL^{\rm SL}$及其应用研究[D]. 西安电子科技大学, 2017.
%\bibitem{BPS}郭楠,贾超. 《信息物理系统白皮书(2017)》解读(上)[J]. 信息技术与标准化, 2017, (4): 36-40.

\bibitem{JH}姜宏. 信息物理融合系统概述[J]. 电脑知识与技术, 2011, 35(7): 9266-9267.

\bibitem{GXS}郭庆来,辛蜀骏,孙宏斌等. 电力系统信息物理融合建模与综合安全评估:驱动力与研究构想[J]. 中国电机工程学报, 2016, 36(6): 1481-1489,1761.

\bibitem{ZS}张曙. 工业4.0和智能制造[J]. 机械设计与制造工程, 2014, 43(8): 1-5.

\bibitem{GXH}管晓宏. 智能时代的信息物理融合系统[J]. 网信军民融合, 2020, (1): 14-17.

\bibitem{ZHUYS}朱永森. 智能系统在城市道路管理中的应用[J]. 科学技术创新, 2019, (16): 125-126.

\bibitem{HOARE}HOARE C A R. Communicating Sequential Processes[M]. New York: Prentice Hall, 1985.

\bibitem{MILNER}MILNER R. A Calculus of Communicating Systems[M]. Berlin: Springer-Verlag, 1980.

\bibitem{PONSE}PONSE A, VERHOEF C, DRS S F M V V. Algebra of Communicating Processes[J]. Theoretical Computer Science, 1997, 37(85): 77–121.

%\bibitem{SDJ}Schneider S, Davies J, Jackson D M, etal. Timed CSP: Theory and Practice[C] //Research and Education in Concurrent Systems Workshop. New York: Springer, 1991: 640–675.
\bibitem{ReedR}REED G M, ROSCOE A W. A Timed Model for Communicating Sequential Processes[J]. Theoretical Computer Science, 1988, 58(1): 249–261.

\bibitem{WY}WANG Y. CCS + Time = An Interleaving Model for Real Time Systems[C] //International Colloquium on Automata, Languages, and Programming. NewYork: Springer, 1991: 217–228.

\bibitem{Lamport}LAMPORT L. Time, Clocks, and the Ordering of Events in a Distributed System[J]. Communications of the ACM, 1978, 21(7): 558-565.

\bibitem{CYW}陈艳文. 分布式系统的 时间化通信行为模型[D]. 华东师范大学, 2014.

\bibitem{DSP}PARKER D S, POPEK G J, RUDISIN G, et al. Detection of Mutual Inconsistency in Distributed Systems[J]. Software Engineering, IEEE Transactions on, 1983, (3): 240–247.

\bibitem{GA}GENE T W, ARTHUR J B. Efficient Solutions to the Replicated Log and Dictionary Problems[J]. Operating Systems Review, 1986, 20(1): 57– 66.

\bibitem{LColin}COLIN J F. Logical Time in Distributed Computing Systems[J]. Computer, 1991, 24(8): 28-33.

\bibitem{FR}FR{\'E}D{\'E}RIC B, ROBERT D S. The Esterel Language[J]. Proceedings of the IEEE, 1991, 79(9): 1293–1304.

\bibitem{AM}ANDR{\'E} C, MALLET F. Clock Constraints in UML/MARTE CCSL[J]. Projet Aoste Rapport de recherche Num-6540, 2008.

\bibitem{Mallet}MALLET F. Clock Constraint Specification Language Specifying Clock Constraints with UML/MARTE[J]. Innovations in Systems and Software Engineering, 2008, 4(3): 309-314.

\bibitem{He}HE J. A Clock-Based Framework for Construction of Hybrid Systems[C] //In: Proceedings of International Colloquium on Theoretical Aspects of Computing, 2013: 22-41.

\bibitem{PNUELI}PNUELI A. The Temporal Logic of Programs[C] //Symposiumon Foundations of Computer Science.Los Alamitos: IEEE Computer Society, 1977: 46–57.

\bibitem{CLARKE}CLARKE E M, EMERSON E A. Design and Synthesis of Synchronization Skeletons using Branching Time Temporal Logic[C] //Logics of Programs. Berlin: Springer Berlin Heidelberg, 1982: 52–71.

\bibitem{EMERSONH}EMERSON E A, HALPERN J Y. “Sometimes” and “Not Never” Revisited: On Branching Versus Linear Time Temporal Logic[J]. ACM, 1986, 33(1): 151-178. 

\bibitem{Alur1}ALUR R, COURCOUBETIS C, DILL D L. Model-Checking in Dense Real-Time[J]. Information and Computation, 1993, 104(1): 2-34.

\bibitem{Alur2}	ALUR R, HENZINGER T A. A really temporal logic[J]. Journal of the ACM, 1994, 41(1): 181-204.

\bibitem{MALLETF}MALLET F. Clock Constraint Specification Language: Specifying Clock Constraints with UML/MARTE[J]. Innovations in Systems and Software Engineering, 2008, 4(3): 309–314.

\bibitem{CLZhou}CHEN B, LI X, ZHOU X. Model Checking of MARTE/CCSL Time Behaviors Using Timed I/O Automata[J]. Journal of Systems Architecture-Embedded Systems Design, 2018, 88: 120–125.

\bibitem{CYYu}CHEN X, YIN L, YU Y, et al. Transforming Timing Requirements into CCSL Constraints to Verify Cyber-Physical Systems[C] //International Conference on Formal Engineering Methods(ICFEM). NewYork: Springer, 2017: 54–70.

\bibitem{HEKL}何康力. 物联网交互系统的量化验证方法研究[D]. 华东师范大学, 2018.

\bibitem{ZHANGYR}张元睿. 面向同步系统的 时钟约束动态逻辑系统研究[D]. 华东师范大学, 2019.

\bibitem{Kazuhiro}KAZUHIRO O, KOKICHI F. Proof Scores in the OTS/CafeOBJ Method[C] //International Conference on Formal Methods for Open Object-Based Distributed Systems FMOODS 2003, Springer, 2003: 170-184.

\bibitem{CBo}陈波,李夫明. 形式化验证方法浅析[J]. 电脑知识与技术, 2019, 15(34): 239-240,250.

\bibitem{Holzmann}HOLZMANN G. The SPIN Model Checker: Primer and Reference Mannual[M]. Addision-Wesley Professional, 2003.

\bibitem{KPW}KIM G L, PAUL P, WANG Y. UPPAAL in a Nutshell[J]. International Journal on Software Tools for Technology Transfer, 1997, 1(1-2): 134-152.

\bibitem{Kenneth}KENNETH L M. Symbolic Model Checking: An Approach to the State Explosion Problem[M]. Kluwer Academic Publishers, 1993.

\bibitem{LTJiao}栾天骄. 实时系统规范语言STeC的Maude语义模型和静态分析设计及其工具实现[D]. 华东师范大学, 2013.

\bibitem{WuH}WU H, CHEN Y, ZHANG M. On Denotational Semantics of Spatial-Temporal Consistency Language -- STeC[C] //International Symposium on Theoretical Aspects of Software Engineering. IEEE Computer Society, 2013: 113-120.

\bibitem{WLH}WANG N, LIU D, HE K. A Formal Description for Protocols in WSN Based on STeC Language[C] //International Conference on Computer Science \& Education, 2013: 921–924.

\bibitem{LCW}栾天骄,陈仪香,王江涛. 实时系统规范语言STeC的Maude重写系统[J]. 计算机工程, 2013, 39(10): 57–62.

\bibitem{ZHANGCH}ZHANG Y, MALLET F, CHEN Y. Timed Automata Semantics of Spatial-Temporal Consistency Language STeC[C] //Theoretical Aspects of Software Engineering Conference, 2014: 201–208.

\bibitem{ZHCHEN}纪政,李慧勇,陈仪香. 基于STeC-Stateflow转换系统的实时系统仿真与验证方法[J]. 计算机应用研究, 2014, 31(2): 448–453.

\bibitem{HEKZHANG}HE K, CHEN Y, ZHANG M, et al. PSTeC: A Location-Time Driven Modelling Formalism for Probabilistic Real-Time Systems[C] //International GI/ITG Conference on Measurement, Modelling, and Evaluation of Computing Systems and Dependability and Fault Tolerance, 2016: 77–91.

\bibitem{WANGSZY}WANG S, ZHANG Y, YANG Z, et al. A Graphical Hierarchical CPS Architecture[C] //International Symposium on System and Software Reliability, 2017: 97–105.

\bibitem{BAIERK}BAIER C, KATOEN J, et al. Principles of Model Checking: Vol 26202649[M]. MIT press Cambridge, 2008.

\bibitem{YZHua}杨志华. 基于STeC的空天地一体化地球观测的验证与仿真[D]. 华东师范大学, 2016.


\end{thebibliography}
