\vspace{-2.5cm}
\chapter*{\zihao{2}\heiti{摘~~~~要}}
\vspace{-1cm}

\setlength{\baselineskip}{25pt}
智慧交通是智慧城市的一个重要研究领域,而面向应急车辆的信号控制方法是智慧交通的一个研究热点,其研究成果能够推动缩短应急车辆的救援时间,对挽救生命和减少财产损失至关重要。传统实现应急车辆快速通行的方法大多是通过传感器或人工控制的方式为应急车辆提供绿灯信号,使得应急车辆顺利通过交叉路口而无需停车。随着汽车保有量的增加,分布式交通信号控制方法成为了智慧交通的新兴技术。本文提出了一种面向应急车辆优先通行的交通信号灯智能控制方法,基于分布式智能交通灯实现应急车辆不停车快速地通过交通灯路口,为应急车辆提供一条能够高速行驶的一路畅通的“绿波带”。本文主要从一下三个方面进行了研究:
%随着中国城市化的高速发展,中国城市车辆保有量也日益增加,智慧交通成为了学术界和工业界的一个重要研究领域。然而交通拥堵导致交通网络应急资源配置效率降低,应急车辆在道路中被堵的问题越来越突出,应急车辆的快速通行成为了智慧交通的一个研究热点,其研究成果能够推动缩短应急车辆的救援时间,对挽救生命和减少财产损失至关重要。传统实现应急车辆快速通行的方法大多是在交叉路口为应急车辆提供绿灯信号(工作人员或是传感器),从而使得应急车辆顺利通过交叉路口而无需停车。而应急车辆在行驶过程中受到道路中其他车辆的影响,限制了应急车辆的高速行驶,从而导致其旅行时间延长。本文着重关注道路中其他车辆对应急车辆造成影响,研究应急车辆、道路中其他车辆以及交通信号灯之间的关系,从应急车辆快速通行出发,设计一种面向应急车辆优先通行的交通信号灯智能配置方法,为应急车辆提供一条能够高速行驶的一路畅通的“绿波带”。本文主要从一下三个方面进行了研究:

%首先,构建面向应急车辆优先通行的交通信号灯智能控制系统架构。在此基础上,构建最快路径“绿波带”模型和对应的分布式智能体模型,并设计了交通信号灯智能控制算法,使得交通信号灯在未请求阶段、降低道路饱和度阶段、信号抢占阶段以及恢复交通流阶段过渡。

%首先,构建面向应急车辆最快路径模型和交叉路口分布式智能体模型以及交通信号灯控制模型。在智能体的控制下,交通信号灯在未请求阶段、降低道路饱和度阶段、信号抢占阶段以及恢复交通流阶段过渡。

首先,构建面向应急车辆优先通行的最快路径“绿波带”模型,在此基础上,构建面向应急车辆优先通行的交通信号灯智能控制系统架构,并设计了交通信号灯智能控制算法,使得交通信号灯在未请求阶段、降低道路饱和度阶段、信号抢占阶段以及恢复交通流阶段过渡。

其次,设计面向应急车辆优先的交通信号灯智能控制方法。第一步通过调整交通信号灯绿灯时间,降低应急车辆行驶路径上的道路饱和度,使得应急车辆前方的其他车辆有空间为应急车辆让行,从而应急车辆能够高速行驶。第二步通过非侵入式抢占与侵入式抢占相结合的信号抢占方法为应急车辆提供绿灯信号,使得应急车辆在绿灯状态下快速通过交通灯路口。第三步通过线性规划方法恢复交叉口各方向交通流,以降低应急车辆优先通行对整个交通造成的影响。在第二步和第三步中,本文实时考虑了所有应急车辆经过的路口,预测其排队车辆数目。

最后,本文在城市交通模拟器SUMO中对面向应急车辆优先通行的交通信号灯智能控制方法的可行性及先进性进行了一系列实验。结果表明,与固定时长信号控制方法(fixed-time control method,简称FTCM)相比,本文方法能够有效缩短应急车辆旅行时间的62.85\%;与弹性型号抢占方法(flexible signal preemption  method,简称FSPM)相比,本文方法能够有效缩短应急车辆旅行时间的50.85\%;与应急车辆信号抢占方法(emergency vehicle signal pre-emption,简称EVSP)相比,本文方法能够有效缩短应急车辆旅行时间的11.62\%。此外,实验结果表明本文方法不会对交通路网造成明显影响。


\hspace{-0.5cm}
\sihao{\heiti{关键词:}} 
%\xiaosi{智慧交通系统,交通信号控制,应急车辆,信号抢占,信号配时,排队分析}
\xiaosi{智慧交通系统,交通信号控制,应急车辆,信号抢占,信号配时}