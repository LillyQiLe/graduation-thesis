%\clearpage{\pagestyle{empty}\cleardoublepage}
%\linespread{1.4}\selectfont
\chapter*{附录 A 速查表}
%\addcontentsline{toc}{chapter}{附录}

\begin{table}[H]
	\centering
	\caption{缩写速查表}
	\label{table:abbreviations}
	\begin{tabular}{|c|c|c|}
		\hline
		缩写 & 含义 & 英文释义 \\ \hline
		EV	 &	应急车辆	  &	emergency vehicle \\ \hline
		DRRS &	降低道路饱和度需求程度	&	demand for reducing road saturation  \\ \hline
		ERL  &  应急响应等级	& emergency response level  \\ \hline
		CLRS &  路段拥堵等级	& congestion level of the road section  \\ \hline
		TUL  &  时间紧迫等级	& time urgency level    \\ \hline
		QP   &  二次规划	  & quadratic programming \\ \hline
		LP   &  线性规划 	  & linear programming \\ \hline
		FTCM &  固定时长信号控制方法 & fixed-time control method \\ \hline
		FSPM &  弹性信号抢占方法 & flexible signal preemption method\cite{min} \\ \hline
		EVSP &  应急车辆信号抢占 & emergency vehicle signal pre-emption\cite{qin_control_2012} \\ \hline
	\end{tabular}
\end{table}

\begin{center}
	\begin{longtable}{|c|l|}
		\caption{符号速查表} 
		\label{table:symbols} \\
		\hline
		符号 & 含义 \\ \hline
		${I}$ & 交叉口集和 \\ \hline
		${E}$ & 路段集和 \\ \hline
		${AGT}$ & 智能体集和 \\ \hline
		${I_i}$ & 第${i}$个交叉口 \\ \hline
		${AGT_i}$ & 第${i}$个智能体 \\ \hline
		${t_i}$ & 预计到达第${i}$个交叉口的时间 \\ \hline
		${v}$ & 应急车辆的平均速度 \\ \hline
		${s_i^t}$ & 应急车辆到达第${i}$个交叉口的距离 \\ \hline
		${\delta}$ & 时间偏差 \\ \hline
		${T_i}$ & 初始信号周期长度 \\ \hline
		${{T_i}^\prime}$ & 新周期长度 \\ \hline
		${L_i^t}$ & 表时刻${t}$到新周期${{T_i}^\prime}$第一次开始时间的间隔 \\ \hline
		${n_i}$ & 交叉口${I_i}$信号灯从${t}$时刻到${t_i}$时间范围内完整新周期${{T_i}^\prime}$的个数 \\ \hline
		${P_i}$ & ${n_i}$个新周期结束到${t_i}$的时间间隔 \\ \hline
		${Q_i}$ & 清空应急车辆入口车道方向的排队车辆所需时间 \\ \hline
		${N_i^t}$ & \tabincell{l}{${t}$时刻交叉口${I_i}$在应急车辆入口车道方向的排队车辆数目} \\ \hline
		${S_i}$ & 交叉口${I_i}$应急车辆入口车道方向的通行能力 \\ \hline
		${A_i}$ & 交叉口${I_i}$应急车辆入口车道方向的车辆到达率 \\ \hline
		${\Delta{w_i}}$ & 代价时间 \\ \hline
		${T_{lost}}$ & 车辆启动损失时间 \\ \hline
		${SIT}$ & 保证该交叉口行车安全的安全时间间隔 \\\hline
		${YT}$ & 普通车辆为应急车辆让行所需的时间 \\ \hline
		${\Delta{i}}$ & \tabincell{l}{表示交叉口${v_i}$的应急车辆入口车道是否存在排队车辆,若应急车辆
			\\入口车道不存在排队车辆,其值为0,若存在,值为1} \\ \hline
		${\beta}$ & 目标函数中平衡总周期变化与单相时长变化的参数 \\ \hline
		${PN}$ &代表周期中信号相位个数 \\ \hline
		${TP}$ & 目标相位 \\ \hline
		${g_i^k}$ & 交叉口${I_i}$原本周期中第${k}$信号相位时长 \\ \hline
		${{g_i^k}^\prime}$ & 交叉口${I_i}$新周期中第${k}$信号相位时长 \\ \hline
		${{T_i}^{max}}$ & 交叉口${I_i}$交通信号灯信号控制周期的上界 \\ \hline
		${{T_i}^{min}}$ & 交叉口${I_i}$交通信号灯信号控制周期的下界 \\ \hline
		${\tau_i^{max}}$ & 交叉口${I_i}$单相位时长的上界 \\ \hline
		${\tau_i^{min}}$ & 交叉口${I_i}$单相位时长的下界 \\ \hline
		${YR_i}$ & 交叉口${I_i}$黄灯与全红时长 \\ \hline
		${{LJT}_i^{TP}}$ & 交叉口${I_i}$的交通信号灯最迟跳转到目标相位${TP}$的时刻 \\ \hline
		${PST_i}$ & ${{LJT}_i^{TP}}$在原信号控制策略中所处相位开始时刻 \\ \hline
		${D_i}$ & 若信号灯在${{LJT}_i^o}$时刻跳转到目标相位,被抢占相位持续的时间 \\ \hline
		${{Jump}_i^{TP}}$ & 交叉口${I_i}$智能体控制交通信号灯跳转到目标相位${TP}$的时刻 \\ \hline
		${G_{minS}}$ & 保证行车安全的最短绿灯时间 \\ \hline
		${AVE_i^k}$  & 交叉口${I_i}$在第${k}$相位绿灯方向的平均排队车辆数 \\ \hline
		${x_i^k}$ & 交叉口${I_i}$的交通信号灯在恢复阶段第${k}$相位的时长 \\ \hline
		${N_i^k}$ & \tabincell{l}{恢复阶段开始时交叉口${I_i}$交通信号灯${k}$相位绿灯方向排队车辆数较大
			\\一边车道聚集的排队车辆数} \\ \hline
		${A_i^k}$ & 交叉口${I_i}$第${k}$相位绿灯方向车道的到达率 \\ \hline
		${S_i^k}$ & 交叉口${I_i}$交通信号灯${k}$相位绿灯方向车道的饱和流率 \\ \hline
		${EQN_i^k}$ & 第${i}$个交叉口第${k}$相位预计排队车辆数 \\ \hline
		${VC_i^k}$ & 第${i}$个交叉口第${k}$相位在该相位时间内能够清除的排队车辆数 \\ \hline
	\end{longtable}
\end{center}

\begin{center}
	\begin{longtable}{|c|c|c|c|c|}
		\caption{目标相位延长时间速查表} 
		\label{table:fulu_extendtime} \\
		\hline
		\tabincell{c}{应急救援等级\\(ERL)} & \tabincell{c}{路段拥堵等级\\(CLRS)}	& \tabincell{c}{时间紧迫等级\\(TUL)} & \tabincell{c}{降低道路\\饱和度程度\\(DRRS)} & \tabincell{c}{延长时间\\(s) }\\ \hline
		特别重大(I级响应) & 严重拥堵(I)   &    严重紧急(I)      &      1			 & 40  \\  \hline
		特别重大(I级响应) & 严重拥堵(I)   &	 比较紧急(II)	 &  1.223912143	     & 40  \\ \hline
		特别重大(I级响应) & 严重拥堵(I)   &	 一般紧急(III)   &  	1.377465866	 & 40  \\ \hline
		特别重大(I级响应) & 中度拥堵(II)  &	 严重紧急(I)	 &  1.521295065	     & 30  \\ \hline
		特别重大(I级响应) & 中度拥堵(II)  &	 比较紧急(II)    &  	1.861931503	 & 30  \\ \hline
		特别重大(I级响应) & 中度拥堵(II)  &	 一般紧急(III)   &  	2.095532024	 & 20  \\  \hline
		特别重大(I级响应) & 轻微拥堵(III) &	 严重紧急(I)     &  	1.944471112	 & 30  \\  \hline
		特别重大(I级响应) & 轻微拥堵(III) &	 比较紧急(II)    &  	2.379861806	 & 20  \\ \hline
		特别重大(I级响应) & 轻微拥堵(III) &	 一般紧急(III)   &  	2.678442585	 & 10  \\ \hline
		特别重大(I级响应) & 畅通(IV)		 &     严重紧急(I)     &  2.314338674      & 20  \\ \hline
		特别重大(I级响应) & 畅通(IV)		 &     比较紧急(II)    &  	  2.832547207  & 10  \\ \hline
		特别重大(I级响应) & 畅通(IV)		 &    一般紧急(III)    &    3.187922526    & 0   \\ \hline
		重大(II级响应)  &   严重拥堵(I)  &	 严重紧急(I)     &  1.074078919	     & 40  \\ \hline
		重大(II级响应)  &   严重拥堵(I)  &	 比较紧急(II)	 &  1.314578231	     & 40  \\ \hline
		重大(II级响应)  &   严重拥堵(I)  &	 一般紧急(III)   &  	1.479507048	 & 30  \\ \hline
		重大(II级响应)  &   中度拥堵(II) &	 严重紧急(I)	 &  1.633990958	     & 30  \\ \hline
		重大(II级响应)  &   中度拥堵(II) &	 比较紧急(II)    &  	1.999861375	 & 20  \\ \hline
		重大(II级响应)  &   中度拥堵(II) &	 一般紧急(III)   &  	2.25076677	 & 20  \\ \hline
		重大(II级响应)  &   轻微拥堵(III)&	 严重紧急(I)     &  	2.088515429	 & 20  \\ \hline
		重大(II级响应)  &   轻微拥堵(III)&	 比较紧急(II)    &  	2.556159395	 & 10  \\ \hline
		重大(II级响应)  &   轻微拥堵(III)&	 一般紧急(III)   & 	2.876858715	     & 10  \\ \hline
		重大(II级响应)  &   畅通(IV)     &	  严重紧急(I)	   &   2.48578238      & 20  \\ \hline
		重大(II级响应)  &   畅通(IV)     &	  比较紧急(II)	   &    3.042379241    & 10  \\ \hline
		重大(II级响应)  &   畅通(IV)     &	  一般紧急(III)    &  	  3.424080379  & 0   \\ \hline
		较大(III级响应) &  严重拥堵(I)   &   严重紧急(I)	      &      1.119930833 & 40  \\ \hline
		较大(III级响应) &  严重拥堵(I)   &   比较紧急(II)      &  	1.370696946	 & 40  \\ \hline
		较大(III级响应) &  严重拥堵(I)   &   一般紧急(III)     &  	1.542666495	 & 30  \\ \hline
		较大(III级响应) &  中度拥堵(II)  & 	严重紧急(I)	      &      1.703745249 & 30  \\ \hline
		较大(III级响应) &  中度拥堵(II)  & 	比较紧急(II)      &  	2.085234499	 & 20  \\ \hline
		较大(III级响应) &  中度拥堵(II)  & 	一般紧急(III)     & 	2.346850925	 & 20  \\ \hline
		较大(III级响应) &  轻微拥堵(III) &	严重紧急(I)	      &      2.177673152 & 20  \\ \hline
		较大(III级响应) &  轻微拥堵(III) &	比较紧急(II)      &  	2.665280615	 & 10  \\ \hline
		较大(III级响应) &  轻微拥堵(III) &	一般紧急(III)     &  	2.999670434	 & 10  \\ \hline
		较大(III级响应) &  畅通(IV)      &	 严重紧急(I)	    &  2.591899239     & 10  \\ \hline
		较大(III级响应) &  畅通(IV)      &	 比较紧急(II)	    &   3.172256952    & 0   \\ \hline
		较大(III级响应) &  畅通(IV)      &	 一般紧急(III)	    &    3.57025273    & 0   \\ \hline
		一般(IV级响应)  &   严重拥堵(I)  &	严重紧急(I)	      &    1.153645523	 & 40  \\ \hline
		一般(IV级响应)  &   严重拥堵(I)  &	比较紧急(II)	  &  1.411960765	 & 40  \\ \hline
		一般(IV级响应)  &   严重拥堵(I)  &	一般紧急(III)	  &  1.58910733	     & 30  \\ \hline
		一般(IV级响应)  &   中度拥堵(II) &  严重紧急(I)	      &      1.755035241 & 30  \\ \hline
		一般(IV级响应)  &   中度拥堵(II) &  比较紧急(II)       &  	2.148008943	 & 20  \\ \hline
		一般(IV级响应)  &   中度拥堵(II) &  一般紧急(III)      &  	2.417501138	 & 20  \\ \hline
		一般(IV级响应)  &   轻微拥堵(III)&  严重紧急(I)	      &      2.243230394 & 20  \\  \hline
		一般(IV级响应)  &   轻微拥堵(III)&  比较紧急(II)       &  	2.745516919	 & 10  \\ \hline
		一般(IV级响应)  &   轻微拥堵(III)&  一般紧急(III)      &  	3.089973297  & 0   \\ \hline
		一般(IV级响应)  &   畅通(IV)     &  严重紧急(I)        &  	  2.669926451  & 10  \\ \hline
		一般(IV级响应)  &   畅通(IV)     &  比较紧急(II)       &  	  3.267755404  & 0   \\ \hline
		一般(IV级响应)  &   畅通(IV)     &  一般紧急(III)      &  	  3.677732551  & 0   \\ \hline
	\end{longtable}
\end{center}

\chapter*{附录 B 调查问卷}
%从零开始编号
\setcounter{table}{0}
%定义编号格式,在数字序号前加字符“A"
\renewcommand{\thetable}{B\arabic{table}}
\renewcommand\thefigure{\Alph{section}\arabic{figure}}

\renewcommand\thefigure{B\arabic{figure}}
\setcounter{figure}{0}    


为了获取应急响应等级(ERL)、路段拥堵等级(CLRS)以及时间紧迫等级的权重关系(TUL),本文采用调查问卷的形式来收集各属性之间的比重。调查问卷的填写遵循Santy 1-9标度方法,如表\ref{table:santy}所示。用户可以通过点击链接进行填写,链接地址为:\href{https://wj.qq.com/s2/10054077/2a68/}{https://wj.qq.com/s2/10054077/2a68/}。也可以扫描如图\ref{fig:qrcode}所示的二维码进行填写。

\begin{figure}[h]
	\centering
	\includegraphics[width=0.3\linewidth]{figures/qrcode}
	\caption{调查问卷二维码}
	\label{fig:qrcode}
\end{figure}

\begin{table}[H]
	\centering
	\caption{Santy1-9标度方法}
	\label{table:santy}
	\begin{tabular}{|c|c|}
		\hline
		标度 & 含义\\
		\hline
		1 &	表示两个因素相比,同样重要 \\ \hline
		3 &	表示两个因素相比,一个因素比另一个因素稍微重要 \\ \hline
		5 &	表示两个因素相比,一个因素比另一个因素明显重要 \\ \hline
		7 &	表示两个因素相比,一个因素比另一个因素强烈重要 \\ \hline
		9 &	表示两个因素相比,一个因素比另一个因素极端重要 \\ \hline
		2,4,6,8 &	上述两相邻判断的中值 \\ \hline
		倒数 &	${j}$与${i}$比较的判断${a_j^i = \dfrac{1}{a_i^j}}$ \\ \hline
	\end{tabular}
\end{table}

下面展示收集到的一份调查问卷,被调查人从事交通方面研究3年,联系方式为467171232@qq.com。

\begin{enumerate}
	\item 题目: 路段拥堵等级相对于应急响应等级?
	
			答:	5 明显重要
	\item 题目: 时间紧迫等级相对于应急响应等级?
	
			答:	3 稍微重要
	\item 题目: 路段拥堵等级相对于时间紧迫等级?
	
			答:	3 稍微重要
\end{enumerate}








