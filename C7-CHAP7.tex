\chapter{总结和展望}

\section{总结}
本文提出了一种面向应急车辆优先通行的交通信号灯智能控制方法,通过按需降低道路饱和度及普通车辆让行、为应急车辆提供信号抢占以及恢复交通流三个阶段保障应急车辆一路快速绿灯通行。第一阶段降低道路饱和度根据应急响应等级、路段拥堵等级以及时间紧迫等级得出目标相位绿灯延长时间。为确保应急车辆到达交通路口时能快速通过交叉口,第二阶段结合侵入式抢占方法与非侵入式抢占方法为应急车辆提供绿灯指引,使得应急车辆能够快速通过交叉路口而不停止。第三阶段通过重新规划信号周期与交通路口各相位绿灯时间,调节交叉路口各入口方向交通流量,降低应急车辆优先对整个交通造成的影响。实验结果表明,与固定时长信号控制方法、Min等人\cite{min}提出的弹性信号抢占方法和Qin等人\cite{qin_control_2012}提出的应急车辆信号抢占方法相比,本文的控制方案能够帮助提升应急车辆的速度,体现应急车辆不停止的“绿波带”效应,缩短应急车辆旅行时间效果更明显。此外,本文的信号恢复策略还能够降低应急车辆优先对整个交通造成的负面影响。

本文做出的贡献如下:
\begin{enumerate}
	\item 本文构建面向应急车辆优先通行的最快路径“绿波带”模型,和分布式智能体模型和面向应急车辆优先通行的交通信号灯智能控制系统架构,并设计了交通信号灯智能控制算法,提出了交通信号灯在未请求阶段、降低道路饱和度阶段、信号抢占阶段以及恢复交通流阶段过渡的方法。
	\item 本文提出了一个新颖的交通信号灯智能控制方法。首先为应急车辆降低道路饱和度,使得普通车辆为应急车辆让行成为可能,并使得应急车辆能够以更高的速度行驶;其次,本文将非侵入性抢占和侵入性抢占相结合,使得本文的信号控制方法既获得了非侵入式抢占的低副作用性又结合了侵入式抢占的高可靠性,与此同时,弥补了非侵入式抢占的不可靠性和降低了侵入式抢占对整个路网的影响;最后本文创新性地将线性规划方法用于恢复交通,当应急车辆通过交叉口后重新规划信号配时,使得交叉路口各方向交通流恢复到应急车辆请求之前的状态。
	\item 在SUMO中对本文方法进行了模拟实验,结果表明本文方法能够有效降低应急车辆的旅行时间,并有效缓解应急车辆优先对整个交通流造成的干扰。
	%\item 本文实时考虑了所有应急车辆即将通过的路口,预测其排队车辆数目,并确保应急车辆到达时前方没有排队车辆。
\end{enumerate}

%首先,构建面向应急车辆优先通行的最快路径“绿波带”模型,在此基础上,构建分布式智能体模型和面向应急车辆优先通行的交通信号灯智能控制系统架构,并设计了交通信号灯智能控制算法,使得交通信号灯在未请求阶段、降低道路饱和度阶段、信号抢占阶段以及恢复交通流阶段过渡。

%其次,设计面向应急车辆优先的交通信号灯智能控制方法。第一步通过调整交通信号灯绿灯时间,降低应急车辆行驶路径上的道路饱和度,使得应急车辆前方的其他车辆有空间为应急车辆让行,从而应急车辆能够高速行驶。第二步通过非侵入式抢占与侵入式抢占相结合的信号抢占方法为应急车辆提供绿灯信号,使得应急车辆在绿灯状态下快速通过交通灯路口。第三步通过线性规划方法恢复交叉口各方向交通流,以降低应急车辆优先通行对整个交通造成的影响。

%最后,本文在城市交通模拟器SUMO中对面向应急车辆优先通行的交通信号灯智能控制方法的可行性及先进性进行了一系列实验。结果表明,与固定时长信号控制方法(fixed-time control method,简称FTCM)相比,本文方法能够有效缩短应急车辆旅行时间的62.85\%;与弹性型号抢占方法(flexible signal preemption  method,简称FSPM)相比,本文方法能够有效缩短应急车辆旅行时间的50.85\%;与应急车辆信号抢占方法(emergency vehicle signal pre-emption,简称EVSP)相比,本文方法能够有效缩短应急车辆旅行时间的11.62\%。此外,实验结果表明本文方法不会对交通路网造成明显影响。

\section{展望}
本文未来的工作包括,但不限于:
\begin{enumerate}
	\item {到达率预测:} 精确的到达率能够帮助我们更加精准地预测排队车辆的数目,从而获得更加准确的清空排队车辆所需的时间,加强本文的抢占策略;
	
	\item {应急车辆的速度分析:} 应急车辆与普通车辆不同,分析应急车辆的速度有助于预测应急车辆的旅行时间,有助于提高抢占的成功率;
	
	\item {面向多辆应急车辆的抢占请求:} 考虑到一些极端情况,例如在较大的公共安全事故中,多辆应急车辆同时向交叉口发出绿灯请求,此时需要设计面向多辆应急车辆的信号控制方案。
	
	\item {延长时间优化:} 本文在降低道路饱和度阶段,延长的绿灯时间不够精确,有可能导致绿灯浪费或者绿灯时间不够。
	
\end{enumerate}