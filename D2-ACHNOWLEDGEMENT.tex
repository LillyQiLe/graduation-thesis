{\kaishu
	\chapter*{致\qquad 谢}
	从2019年9月进入软件学院至今,这三年的时光转瞬即逝,在华东师范大学的研究生生活无疑是我人生中宝贵的财富之一,在这三年的时间中我学会了如何去发现问题,独立思考问题,如何将所学知识活学活用,让我对自己有了更加深刻的了解,如何更好的规划未来的人生道路。在毕业论文即将完成之际,我要对每一位支持我,学习上鼓励我、教导我,生活上关心我的人表示衷心的感谢。
	
	首先,我要感谢软件学院的所有老师们,老师们具有一流的科研水平以及极高科研热情,带领我们以当下最先进,最前沿的眼光来审视当下各种问题,如今华东师范大学的软件学院位列软科全国软件学院排名第二名,这来之不易的成果无一不归结于软件学院所有老师的辛勤付出与莘莘学子的艰苦意志。陈仪香老师是嵌入式系统、软硬件协同设计、异构计算、信息物理融合系统、实时系统、形式化、分布式系统以及人工智能方面的专家,在学术界很有名气,研究成果如雨后春笋。感谢陈老师在研究生期间对我的教导,陈老师强大的专业背景让我认识到了智慧城市的魅力之处,总能从陈老师的耐心讲解中找到正确的求解思路,感谢陈老师在研究室期间对我的信任与包容,不管是在个人发展以及学术研究上,陈老师总能给我最大的帮助,理解与关心。
	
	其次感谢实验室陪伴我走过研究生生涯的同门,三年的时间,在这里见证了大家的喜怒哀乐,感谢许巾一博士在我刚来实验室时对我的学术指导,感谢吴玮琳、刘鹏晨、王蒙、丁立诚、任达万、刘晗、冒雯雯、叶文韬、蒋清源、石昊、唐付奇、聂奇隆、岳泽龙、陈学毅、侯雪城、陈新宇等在实验室学习的师兄姐弟妹们,感谢嵌入式系所有的同学们。还要感谢这三年来包容我,陪伴我度过最美好的研究生生活的舍友们。我还要感谢我的朋友,感谢许鹏飞、钟鸣、陈文茜,在我遇到困难时给我鼓励和帮助,感谢他们给我的爱与尊重。
	
	此外,我想感谢所有软件学院的老师和朋友,感谢陪我走过三年研究生生活的的软件学硕班,在这个班集体中我遇到了积极向上,执着努力热爱生活的一帮程序员么。
	
	最后,感谢我的父母,他们是我背后最坚强的后盾。他们教会了我成为一个正直,勤奋,有担当的人,总是默默的支持我的任何想多的事情,无私的付出。希望在将来不会辜负他们对我的期望,对得起父母的养育之恩。

\vspace{0.2cm} \hspace{11.5cm}
钟力
\hspace{10.6cm} 

~~~~~~~~~~~~~~~~~~~~~~~~~~~~~~~~~~~~~~~~~~~~~~~~~~~~~~~~~~~~~~~~~~~~~~~~~~~~~~~~~~~~~~~~~~~~~~~~~~~~~~二〇二二年五月 }
