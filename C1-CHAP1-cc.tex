\chapter{绪\hskip 0.4cm 论}
\label{chap1}
智能系统是当前的研究热点,其逻辑系统的研究也至关重要。本章对论文的研究背景与意义进行阐述,分析国内外研究现状,并简述本论文的主要工作和论文组织架构。

\section{研究背景与意义}
%随着计算机技术的快速发展,我们的社会生活正在全面进入智能时代,智能系统的应用已渗透到人们生产生活的方方面面。自1956年人工智能的概念提出以来,伴随着大数据、云计算、物联网等技术的发展,也不断地推动着智能系统的发展\upcite{TTN}。信息物理融合系统(Cyber Physical Systems, CPS)就是一个融合了计算过程和物理过程,并通过嵌入式计算机与网络实现两者之间的协作和融合的多维复杂智能系统\upcite{LXLL}。随着智能系统的发展,它引起了国内外学术界和工业界的广泛关注,不同的专家学者从不同的角度对其进行了定义,中国科学院院士何积丰的定义:“CPS是在环境感知的基础上,深度融合了计算、通信和控制能力的可控、可信、可扩展的网络化物理设备系统,它通过计算进程和物理进程相互影响的反馈循环实现深度融合和实时交互来增加或扩展新的功能,以安全、可靠、高效和实时的方式监测或者控制一个物理实体”\upcite{Hjf}。

%CPS的核心是“3C”技术,即计算(Computation)、通信(Communication)和控制(Control)这三种技术,通过把这三种技术与嵌入式相结合,把网络通信技术和计算机的大规模与精算能力嵌入到现实对象中,使信息世界和物理世界相互联结,实现了物理对象的自我判断\upcite{LZZ,LFJWQ,ZYW},它的结构图如图\ref{CPS}所示。CPS在安全攸关的领域有着广泛的应用,如航空航天、高速铁路、智能家居、智能医疗等。在这些安全攸关的领域,系统不仅拥有计算、通信、控制的能力,也强调时间和空间的信息,保证系统的时空约束从而保证系统的可靠性与安全性至关重要。

%\begin{figure}[htbp]
%	\centering
%	\includegraphics[width=10cm]{wangpf/CPS.png}
%	\caption{CPS结构图}
%	\label{CPS}
%\end{figure}
%随着计算机技术的快速发展,我们的社会生活正在全面进入智能时代,智能系统的应用已渗透到人们生产生活的方方面面。自1956年人工智能的概念提出以来,伴随着大数据、云计算、物联网等技术的发展,也不断地推动着智能系统的发展\upcite{TTN}。智能系统的典型应用有智能交通、智能制造、智能医疗、智能家居等等,这些都是与人们生产生活息息相关安全攸关的智能系统,因此对智能系统的建模、规约以及验证变得愈发重要。

随着计算机技术的快速发展,我们的社会生活正在全面进入智能时代,智能系统的应用已渗透到人们生产生活的方方面面。自1956年人工智能的概念提出以来,伴随着云计算、物联网、大数据等技术的发展,也不断地推动着智能系统的发展\upcite{TTN}。信息物理融合系统(Cyber Physical Systems, CPS)是对计算过程和物理过程进行融合,并通过嵌入式计算机与网络实现两者之间协作的多维复杂智能系统\upcite{LXLL}\upcite{Hjf},它的核心是“3C”技术,即计算(Computation)、通信(Communication)和控制(Control),通过把这三种技术与嵌入式相结合,把网络通信技术和计算机的大规模与精算能力嵌入到现实对象中,从而使信息世界和物理对象相互联结\upcite{LZZ,LFJWQ,ZYW}。CPS的典型应用有智能交通、智能制造、智能医疗、智能家居等等,这些都是与人们生产生活息息相关安全攸关的智能系统,因此对智能系统的规约以及验证变得愈发重要。

形式化方法是提高系统的可靠性和安全性的常用方法,它通过对系统建立形式化模型,继而使用逻辑推理或者模型检测手段来验证或分析模型的性质\upcite{CEM}。形式化方法在智能系统中有广泛的应用,在具有时空约束的系统里,需要同时对系统的时间和空间信息进行约束,保证系统的时空一致性具有重要意义。例如高铁需要在规定的时间到达规定的站点,否则会导致列车晚点,严重的会波及到更多的列车,给人们的出行造成不便。为了对智能体的时空信息进行建模,陈仪香教授在2012年提出了实时系统规范语言 STeC(Spatio-Temporal Consistency Language)\upcite{Chen},这是一种强调智能体时空一致性的语言,通过对智能体的时间和空间信息同时进行约束,达到保证其时空一致性的效果\upcite{WSZ}。

时钟系统用来规约系统的时间特性,某些安全攸关的智能系统不仅拥有计算、通信和控制的能力,而且同时强调系统中智能体的时间和空间信息,保证系统的时空约束对于保证系统的可靠性与安全性非常关键。物理时钟描述的是具体的时间,逻辑时钟代表了事件发生的先后顺序,在智能系统中需要对逻辑时钟和物理时钟同时进行约束,陈仪香教授在2014年提出了混成时钟的概念以同时规约实时系统的逻辑时钟和物理时钟,该时钟中时空点的集合包含了智能体的时间和空间信息\upcite{ChenY}。

模型检测是一种自动验证有限状态系统性质正确性的技术,将系统的行为用模型规约,将系统所具有的性质用模态逻辑规约,然后验证模型规约是否满足性质规约。系统模型可以是软件系统、硬件系统或软硬件协同系统,规约是由逻辑公式描述的系统的性质,如系统的安全性等。给定系统的模型,能够通过穷尽搜索和自动检查的方式判定该模型能否满足给定的规约,逻辑公式可以对系统的需求以及性质进行规约。随着软件工程和人工智能的发展,逻辑对系统的规约与验证有着重要的作用,对计算机科学产生了深远的影响,因此,对逻辑的研究具有重要的意义,典型的逻辑有描述时间关系的时态逻辑、描述空间关系的空间逻辑以及将时态逻辑和空间逻辑相结合的时空逻辑\upcite{AE,JZ,LXu}。

在安全攸关的智能系统中,对智能体的逻辑时钟和物理时钟同时进行规约以及保证其时空一致性至关重要,本文基于对智能体的物理时钟和逻辑时钟同时进行规约的混成时钟,构建了混成时钟逻辑系统,对智能系统中智能体的性质进行刻画,保证系统的时空一致性,通过提出的混成时钟逻辑模型检测算法以及开发的STeC模型检测工具对智能体的性质进行检测,从而更加有效地提高系统的可靠性和安全性。

\section{国内外研究现状}
%随着我们的社会生活进入智能时代,智能系统也成为了研究的热点。智能制造、智能交通、智能家居、智能电网、智能医疗等领域都取得了丰硕的成果。CPS作为集计算、通信和控制于一体的下一代智能系统\upcite{JH},自2006年由美国国家科学基金会(National Science Fundation, NSF)提出以来,由于其具有很高的科研意义,得到了国内外学者的高度重视,同时,其潜在的商业价值与应用意义,也得到了工业界的广泛关注。我国发布的《国务院关于深化制造业与互联网融合发展的指导意见》中将发展CPS作为加强融合发展基础支持的重要组成部分\upcite{BPS}。在学术界,郭庆来等学者提出了一种用于智能电网的CPS融合建模构想,以分析和评估电力系统中的信息故障\upcite{GXS}。在工业界,CPS的深度融合作为工业4.0的核心\upcite{ZS},智能制造成为一种新的趋势,如在3D打印等一些新型制造技术中,CPS可以通过云计算技术利用云平台连接用户、生产商和服务商,实现制造的智能化。在智能家居方面,通过物联网技术可以将家电和家居环境连接起来\upcite{GXH}。利用智能交通系统能够增强道路通行能力,减小城市交通压力\upcite{ZHUYS}等。
随着我们的社会生活进入智能时代,智能系统也成为了研究的热点。智能制造、智能交通、智能家居、智能电网、智能医疗等领域都取得了丰硕的成果。CPS作为集计算、通信和控制于一体的下一代智能系统\upcite{JH},自2006年由美国国家科学基金会(National Science Fundation, NSF)提出以来,由于其具有很高的科研意义,得到了国内外学者的高度重视,同时,其潜在的商业价值与应用意义,也得到了工业界的广泛关注。在学术界,郭庆来等学者提出了一种用于智能电网的CPS融合建模构想,以分析和评估电力系统中的信息故障\upcite{GXS}。在工业界,CPS的深度融合作为工业4.0的核心\upcite{ZS},智能制造成为一种新的趋势,如在3D打印等一些新型制造技术中,CPS可以通过云计算技术利用云平台连接用户、生产商和服务商,实现制造的智能化。在智能家居方面,通过物联网技术可以将家电和家居环境连接起来\upcite{GXH}。利用智能交通系统能够增强道路通行能力,减小城市交通压力\upcite{ZHUYS}等。

形式化方法是基于数学化的方法对系统进行规约和验证的技术\upcite{JZ}。在高可信的系统中,对系统的建模以及对系统的安全性和可靠性的验证变得至关重要,形式化技术得到了广泛研究与应用。

在计算机科学中,进程代数是对一些并发系统的形式化建模语言的统称,具有代表性的进程代数语言有CSP\upcite{HOARE},CCS\upcite{MILNER},ACP\upcite{PONSE}等。这些语言主要描述系统的功能,但在实时系统中,实时性非常重要,继而带有时间的形式化建模语言被提出,Reed和Roscoe将WAITt加入到进程代数语言CSP中,形成了Timed CSP\upcite{SDJ},Wang Yi将时间变量引入到CCS中,形成了Timed CCS\upcite{WY}等。在一些安全攸关的领域,除了需要考虑时间信息,同时还需要考虑空间信息,据此,陈仪香教授在2012年提出了能刻画时空一致性的实时系统规范语言STeC\upcite{Chen}。

%自动机方面的研究也成果丰硕,它是一种数学计算模型,可以用来设计计算机程序和时序逻辑电路等,典型的自动机有Moore自动机[10]、Mealy自动机[11睿]、Alur和Dill提出的时间自动机(Timed Automata)[23]以及Henzinger和T.A.提出的混成自动机(Hybrid Automata)[24]等。
时钟系统用来规约系统的时间特性,通常有两种时钟:精密计时时钟(即物理时钟)和逻辑时钟。物理时钟是指现实世界中的实际时间,逻辑时钟于1978年由Leslie Lamport提出用来描述分布式系统中事件的发生顺序\upcite{Lamport},根据Lamport的逻辑时钟,接着提出了两个更高级的逻辑时钟(矢量时钟和矩阵时钟)来捕获分布式计算事件之间的因果关系\upcite{CYW,DSP,GA}。然后,它被扩展并用于分布式系统中,以测试通信的正确性\upcite{LColin}。逻辑时钟在同步语言中也有广泛应用,在同步域中,它已被证明适用于任何级别的描述\upcite{FR}。Andr{\'e}和Mallet将时钟定义为五元组$\langle \mathcal{I},\prec,\mathcal{D},\lambda,u\rangle$\upcite{AM}\upcite{Mallet},何积丰院士将时钟定义为无限增加的非负实数序列\upcite{He},陈仪香教授将物理时钟与逻辑时钟相结合定义了混成时钟\upcite{ChenY}来对实时系统的时钟进行描述。

为了对系统的性质进行描述,提出了一些逻辑,如线性时态逻辑LTL\upcite{PNUELI},计算树逻辑CTL\upcite{CLARKE}、CTL*\upcite{EMERSONH}等,接着提出了一些带有时间的逻辑,如TCTL\upcite{Alur1}、TPTL\upcite{Alur2}等。一种基于时钟约束的规约语言CCSL\upcite{MALLETF}也被提出,并得到了广泛的研究\upcite{CLZhou}\upcite{CYYu}。之后用于规范物联网交互系统性质的代价概率时间数据流逻辑pPTDL被提出\upcite{HEKL}。为了将动态逻辑应用于同步系统的建模与验证中,提出了一种基于时钟约束的动态逻辑系统\upcite{ZHANGYR}。

形式化验证方面也得到了广泛的研究,主要有两种方法,一种是基于逻辑推理的定理证明技术,另一种是基于穷尽搜索的模型检测技术。定理证明技术具有很强的逻辑表达能力,可以处理无限的状态空间,避免状态空间爆炸问题,主要的定理证明工具有Coq、STeP、CafeOBJ\upcite{Kazuhiro}\upcite{CBo}等。模型检测主要用于硬件和协议的验证,对软件设计的验证也成为研究热点\upcite{CBo}。主要的模型检测工具有Bell实验室研制的用于对分布式系统进行验证的开源工具SPIN\upcite{Holzmann},基于时间自动机的对实时系统进行建模、验证的工具UPPAAL\upcite{KPW}、主要用于计算机硬件设计的符号模型检测工具SMV\upcite{Kenneth}等。

\section{本文工作与主要贡献}
进入智能时代,智能系统的研究越来越受到人们的关注。在一些安全攸关的智能系统中,智能体的时空信息需要保持一致,并且系统的物理时钟和逻辑时钟都需要规约,如何从保证时空一致性的角度对系统的性质进行刻画并进行验证显得十分重要。本文将围绕此进行研究,主要研究工作如下:

\begin{enumerate}
	\item 为了建立混成时钟逻辑的语法结构,本文对混成时钟系统进行了更加深入的研究,在已有的混成时钟运算和混成时钟关系的基础上提出了一些新的运算和关系以满足系统性质描述的需求,并研究了混成时钟关系的性质,同时给出了具体的实例对混成时钟运算和关系进行解释,最后基于混成时钟运算和关系定义了混成时钟逻辑的语法。
	
	\item 构建了混成时钟逻辑系统的演算理论,首先通过定义STeC设计与HCL公式之间的可满足关系给出了HCL的语义,证明了HCL公式的语义等价关系,接着提出了混成时钟逻辑模型检测算法,并通过案例研究表明了混成时钟逻辑的可用性和模型检测算法的可行性。
	
	\item 为了进行模型检测,本文设计开发了STeC模型检测工具,自动化地实现将STeC模型装换成混成时钟,进而对以混成时钟逻辑公式描述的系统的性质进行验证。
\end{enumerate}

\section{组织结构}
第二章介绍本文研究涉及到的预备知识,包括实时系统规范语言STeC和混成时钟系统。

第三章对混成时钟逻辑系统语法结构的构建进行阐述,首先利用STeC语言对安全攸关的系统进行形式化建模,接着对混成时钟的运算和关系进行介绍,最后基于混成时钟的运算和关系建立混成时钟逻辑的语法结构。

第四章介绍混成时钟逻辑的模型检测,首先介绍混成时钟逻辑系统的演算理论,定义HCL的语义,证明混成时钟逻辑公式的语义等价关系,接着提出混成时钟逻辑模型检测算法,最后进行案例研究。

第五章介绍STeC模型检测工具,包括工具的整体介绍、设计思路以及使用说明,并对高速铁路和航空航天领域智能体的性质进行验证。

第六章对全文进行总结,并展望下一步的工作。




